\documentclass[11pt,a4paper]{report}
\usepackage[utf8]{inputenc}
\usepackage[left=3.5cm, right=2.5cm, top=2.5cm, bottom=2.5cm]{geometry}
\usepackage{setspace}
\usepackage{titlesec}
\usepackage{fancyhdr}
\usepackage{graphicx}
\usepackage{caption}
\usepackage{footmisc}
\usepackage{hyperref}
\usepackage{etoolbox}

% Schriftart und Zeilenabstand
\renewcommand{\familydefault}{\sfdefault} % Schriftart (Arial-ähnlich)
\setlength{\parindent}{0pt} % Kein Absatzeinzug
\setlength{\parskip}{0.5em} % Abstand zwischen Absätzen
\onehalfspacing % Zeilenabstand 1,5

% Überschriftenformate anpassen: Gleiche Schriftgröße wie der Text
\titleformat{\chapter}%[hang]
  {\normalfont\bfseries\normalsize} % Normale Textgröße (normalsize) für Kapitelüberschrift
  {\thechapter.} % Kapitelnummerierung mit Punkt
  {1em} % Abstand zwischen Nummer und Titel
  {}

\titleformat{\section}
  {\normalfont\bfseries\normalsize} % Normale Textgröße (normalsize) für Abschnittsüberschrift
  {\thesection.} % Abschnittsnummerierung mit Punkt
  {1em} % Abstand zwischen Nummer und Titel
  {}

\titleformat{\subsection}
  {\normalfont\bfseries\normalsize} % Normale Textgröße (normalsize) für Unterabschnittsüberschrift
  {\thesubsection.} % Unterabschnittsnummerierung mit Punkt
  {1em} % Abstand zwischen Nummer und Titel
  {}

% Fußnotenformat
\renewcommand{\footnotelayout}{\setstretch{1}} % Zeilenabstand in Fußnoten

% Kopf-/Fußzeilen
\pagestyle{fancy}
\fancyhf{}
\fancyhead[L]{Name der Arbeit}
\fancyhead[R]{\thepage}

% Beschriftungsformat für Tabellen anpassen
\captionsetup[table]{
    labelformat=simple,
    labelsep=colon, % Trennt "Tabelle Nr" und Titel mit einem Doppelpunkt
    name=Tabelle % Festlegen des Präfixes für die Beschriftung
}

% Nummerierung der Tabellen ohne Abschnittsnummer
\renewcommand{\thetable}{\arabic{table}} % Tabellen nur fortlaufend nummerieren

% Beschriftungsformat für Abbildungen anpassen
\captionsetup[figure]{
    labelformat=simple,
    labelsep=colon, % Trennt "Abbildung Nr" und Titel mit einem Doppelpunkt
    name=Abbildung % Festlegen des Präfixes für die Beschriftung
}

% Nummerierung der Abbildungen ohne Abschnittsnummer
\renewcommand{\thefigure}{\arabic{figure}} % Abbildungen nur fortlaufend nummerieren

\usepackage{tocloft} % Für Anpassungen im Inhalts-, Abbildungs- und Tabellenverzeichnis

% Anpassung des Tabellenverzeichnisses
\renewcommand{\cfttablename}{Tabelle} % Präfix "Tabelle" hinzufügen
\renewcommand{\cfttabpresnum}{\cfttablename\enspace} % Präfix vor der Nummer
\renewcommand{\cfttabaftersnum}{:} % Doppelpunkt nach der Nummer
\setlength{\cfttabnumwidth}{6em} % Erhöhte Breite für "Tabelle Nr" im Verzeichnis

% Anpassung des Abbildungsverzeichnisses
\renewcommand{\cftfigurename}{Abbildung} % Präfix "Abbildung" hinzufügen
\renewcommand{\cftfigpresnum}{\cftfigurename\enspace} % Präfix vor der Nummer
\renewcommand{\cftfigaftersnum}{:} % Doppelpunkt nach der Nummer
\setlength{\cftfignumwidth}{6em} % Erhöhte Breite für "Abbildung Nr" im Verzeichnis

% Anpassung der Verzeichnisüberschriften
\renewcommand{\contentsname}{Inhaltsverzeichnis}
\renewcommand{\listfigurename}{Abbildungsverzeichnis}
\renewcommand{\listtablename}{Tabellenverzeichnis}
\renewcommand{\bibname}{Literaturverzeichnis}

\usepackage{tocloft} % Für Anpassungen der Verzeichnisüberschriften

% Schriftgröße für Verzeichnisüberschriften auf 11pt setzen
\renewcommand{\cfttoctitlefont}{\normalfont\normalsize\bfseries} % Inhaltsverzeichnis
\renewcommand{\cftloftitlefont}{\normalfont\normalsize\bfseries} % Abbildungsverzeichnis
\renewcommand{\cftlottitlefont}{\normalfont\normalsize\bfseries} % Tabellenverzeichnis

% Einheitliches Layout für alle Seiten
\fancyhf{} % Setzt Kopf- und Fußzeilen zurück
\fancyfoot[R]{\thepage} % Seitennummer rechts unten
\renewcommand{\headrulewidth}{0pt} % Entfernt die Linie in der Kopfzeile
\renewcommand{\footrulewidth}{0pt} % Entfernt die Linie in der Fußzeile

% Kapitelüberschrift: Gleicher Abstand nach unten wie bei Abschnittsüberschriften
\titlespacing{\chapter}{0pt}{0.5cm}{0.5cm} % Abstand oben und unten 0.5cm

% Abschnittsüberschrift: Gleicher Abstand wie Kapitelüberschrift
\titlespacing{\section}{0pt}{0.5cm}{0.5cm}

% Unterabschnittsüberschrift: Optional anpassen, falls nötig
\titlespacing{\subsection}{0pt}{0.5cm}{0.5cm}

\usepackage[style=apa,backend=biber,lang=ngerman]{biblatex} % APA-Format und deutsche Sprache
\addbibresource{B_Literatur/100_literatur.bib} % Pfad zur Literaturdatenbank
% Variablen für die Arbeit
\newcommand{\ThesisTitle}{Titel des Papers}
\newcommand{\Module}{Lehrveranstaltung}
\newcommand{\AuthorName}{Max Mustermann}
\newcommand{\MatriculationNumber}{123456}
\newcommand{\Supervisor}{Prof. Dr. Beispiel}
\newcommand{\SubmissionDate}{01.01.2024} % Lädt die Variablen aus der Datei variables.tex

\begin{document}

% Sperrvermerk
\chapter*{Sperrvermerk}
Die vorliegende Masterarbeit beinhaltet in den Kapiteln XYZ vertrauliche Informationen der Firma XYZ. 
Diese Kapitel sind daher nur den Gutachtern sowie den Mitgliedern des Prüfungsausschusses zu Prüfungszwecken zugänglich zu machen. Veröffentlichungen und Vervielfältigungen der betroffenen Kapitel sind ohne ausdrückliche Genehmigung des Unternehmens nicht gestattet.
\medskip
Dieser Sperrvermerk gilt XYZ Jahre ab dem Einreichungsdatum der Arbeit beim Prüfungsamt.
\vspace{4cm}
\hfill Rohrbach, den \today \hrulefill
\thispagestyle{empty}

% Deckblatt
\begin{titlepage}
    \raggedright % Flattersatz nur für das Deckblatt
    % Deckblatt
\begin{titlepage}
    \centering
    {\fontsize{20pt}{26pt}\selectfont \textbf{\ThesisTitle}}\\[2cm]
    {\large im Rahmen der Lehrveranstaltung}\\[0cm]
    {\large \textbf{\Module}}\\[1cm]
    {\large\textbf{Executive Education}\\[0cm] MCI | Die Unternehmerische Hochschule®}\\[7cm]
    {\large Betreuer: \textbf{\\[0cm]\Supervisor}}\\[2cm]
     {\large Verfasser:in: \textbf{\\[0cm]\AuthorName \\[0cm]\MatriculationNumber}} \\[2cm]
    {\large Abgabedatum: \\[0cm]\textbf{\SubmissionDate}}\\[2cm]
\end{titlepage} % Einbindung der ausgelagerten Datei
\end{titlepage}
\thispagestyle{empty}

% Einfügen der Eidesstattlichen Erklärung
\newpage
\chapter*{Eidesstattliche Erklärung}
\chaptermark{Eidesstattliche Erklärung}

Hiermit erkläre ich, Ein Autor, dass ich die vorliegende Masterarbeit selbstständig und ohne unerlaubte Hilfe angefertigt, andere als die angegebenen Quellen und Hilfsmittel nicht benutzt und die den benutzten Quellen wörtlich oder inhaltlich entnommenen Stellen als solche kenntlich gemacht habe.

Die Arbeit wurde bisher in gleicher oder ähnlicher Form keiner anderen Prüfungsbehörde vorgelegt und auch nicht veröffentlicht.
\vspace{1cm}

\hfill Rohrbach, den \today \hrulefill
\thispagestyle{empty}

% Einfügen der Kurzfassung/Abstract
\newpage
\chapter*{Kurzfassung/Abstract}

\section*{Kurzfassung}
Diese Masterarbeit behandelt das Thema XYZ und analysiert die Aspekte ABC basierend auf theoretischen Grundlagen und empirischen Erkenntnissen. Ziel der Arbeit ist es, Handlungsempfehlungen abzuleiten, die für DEF von Relevanz sind. Die Ergebnisse zeigen, dass GHI einen entscheidenden Einfluss auf JKL haben und durch geeignete Maßnahmen optimiert werden können.

Die Arbeit liefert einen Beitrag zur aktuellen Forschung im Bereich MNO und kann als Grundlage für weitere Studien dienen.

%\cite{LABEL1,LABEL2,NORM1}

\section*{Abstract}
This master's thesis explores the topic of XYZ and analyzes aspects of ABC based on theoretical foundations and empirical findings. The aim of the study is to derive practical recommendations relevant for DEF. The results demonstrate that GHI significantly impacts JKL and can be optimized through appropriate measures.

The work contributes to current research in the field of MNO and serves as a basis for future studies.

%\cite{LABEL1,LABEL2,NORM1}
\thispagestyle{empty}

\newpage
\thispagestyle{empty}
\cleardoublepage

% Inhaltsverzeichnis mit Seitennummerierung starten
\pagenumbering{Roman} % Römische Zahlen starten
\setcounter{page}{1}  % Startet die Nummerierung bei 1
\tableofcontents
\newpage

% Abbildungsverzeichnis
\listoffigures
\addcontentsline{toc}{chapter}{\listfigurename} % Zum Inhaltsverzeichnis hinzufügen
\newpage

% Tabellenverzeichnis
\listoftables
\addcontentsline{toc}{chapter}{\listtablename} % Zum Inhaltsverzeichnis hinzufügen
\newpage

% Abkürzungsverzeichnis
\chapter*{Abkürzungsverzeichnis}
\addcontentsline{toc}{chapter}{Abkürzungsverzeichnis}
\begin{description}
    \item[AI] Artificial Intelligence
    \item[CRM] Customer Relationship Management
\end{description}
\newpage

% Arabische Seitennummerierung für den Hauptteil
\pagenumbering{arabic}
\setcounter{page}{1} % Startet die arabische Nummerierung bei 1

%######### KAPITEL EINFÜGEN ##############
% Beispielkapitel
\chapter{Einleitung}
Dies ist der Einleitungstext. Ziel der Arbeit ist es...

\section{Hintergrund}
Lorem ipsum dolor sit amet, consetetur sadipscing elitr, sed diam nonumy eirmod tempor invidunt ut labore et dolore magna aliquyam erat, sed diam voluptua. At vero eos et accusam et justo duo dolores et ea rebum. Stet clita kasd gubergren, no sea takimata sanctus est Lorem ipsum dolor sit amet. Lorem ipsum dolor sit amet, consetetur sadipscing elitr, sed diam nonumy eirmod tempor invidunt ut labore et dolore magna aliquyam erat, sed diam voluptua. At vero eos et accusam et justo duo dolores et ea rebum. Stet clita kasd gubergren, no sea takimata sanctus est Lorem ipsum dolor sit amet.
\cite{Muster.2024}

\subsection{Relevanz der Thematik}
Warum ist dies wichtig...
\chapter{Theorie}
\section{Strategisches Management}
Einleitung in das Thema strategisches Management.

% Abbildung
\begin{figure}[h!]
    \centering
   
    \caption{Beispielhafte Abbildung mit Titel.}
    \label{fig:beispiel}
\end{figure}

% Tabelle
\begin{table}[h!]
    \centering
    \caption{Beispieltabelle mit Titel.}
    \label{tab:beispieltabelle}
    \begin{tabular}{|c|c|}
        \hline
        Spalte 1 & Spalte 2 \\
        \hline
        Wert 1 & Wert 2 \\
        \hline
    \end{tabular}
\end{table}
%###### KAPITEL EINFÜGEN ENDE ############

% Erklärung zu generativer KI
\chapter*{Erklärung zu generativer KI und KI-gestützten Technologien}
\addcontentsline{toc}{chapter}{Erklärung zu generativer KI und KI-gestützten Technologien}
In der vorliegenden Praxisarbeit wurde die Unterstützung von generativen KI-Systemen
als Hilfsmittel für die Recherche, die Strukturierung von Inhalten und die Formulierung
von Texten in Anspruch genommen. Der Autor dieser Arbeit war verantwortlich für die
Konzeption, Planung und Durchführung der Forschungsarbeit, einschließlich der Definition
der Forschungsfragen, der Analyse der Daten und der Interpretation der Ergebnisse.

Generative KI-Systeme wurden insbesondere zur Unterstützung in den folgenden Bereichen
eingesetzt:
\begin{itemize}
    \item \textbf{Recherche und Literatursuche:} Generative KI-Systeme halfen bei der Identifikation relevanter Literaturquellen und der Formulierung von Suchstrategien.
    \item \textbf{Ideenentwicklung und Strukturierung:} Die Modelle wurden genutzt, um verschiedene Perspektiven und Ansätze zur Bearbeitung der Themenstellung zu entwickeln und die Struktur der Arbeit zu optimieren.
    \item \textbf{Textformulierung und -paraphrasierung:} Generative KI-Systeme unterstützten bei der Formulierung und Paraphrasierung von Texten, um präzise und verständliche Darstellungen der Ergebnisse und Diskussionen zu gewährleisten.
\end{itemize}

Die finale Entscheidung über die Einbindung von Inhalten, die Interpretation der Ergebnisse
sowie die Schlussfolgerungen der Arbeit oblagen jedoch ausschließlich dem Autor.
Alle durch Generative KI-Systeme generierten Inhalte wurden sorgfältig geprüft, angepasst
und in den Kontext der spezifischen Fragestellung dieser Arbeit integriert. Die Verantwortung
für den gesamten Inhalt und die wissenschaftliche Integrität dieser Arbeit
liegt beim Autor. % Einbindung der ausgelagerten Datei

% Literaturverzeichnis
\addcontentsline{toc}{chapter}{\bibname}
\bibliographystyle{apalike}
\bibliography{B_Literatur/100_literatur} % Pfad zur Literaturdatenbank

% Anhang
\appendix
\chapter{Anhang}
Zusätzliche Materialien.

\end{document}