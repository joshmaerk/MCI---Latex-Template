\documentclass[11pt,a4paper]{report}
\usepackage[utf8]{inputenc}
\usepackage[left=3.5cm, right=2.5cm, top=2.5cm, bottom=2.5cm]{geometry}
\usepackage{setspace}
\usepackage{titlesec}
\usepackage{fancyhdr}
\usepackage{graphicx}
\usepackage{caption}
\usepackage{footmisc}
\usepackage{hyperref}
\usepackage{etoolbox}

% Schriftart und Zeilenabstand
\renewcommand{\familydefault}{\sfdefault} % Schriftart (Arial-ähnlich)
\setlength{\parindent}{0pt} % Kein Absatzeinzug
\setlength{\parskip}{0.5em} % Abstand zwischen Absätzen
\onehalfspacing % Zeilenabstand 1,5

% Überschriftenabstände
\titlespacing\chapter{0pt}{0pt}{1cm}
\titlespacing\section{0pt}{1cm}{0.5cm}
\titlespacing\subsection{0pt}{0.5cm}{0.3cm}

% Fußnotenformat
\renewcommand{\footnotelayout}{\setstretch{1}} % Zeilenabstand in Fußnoten

% Kopf-/Fußzeilen
\pagestyle{fancy}
\fancyhf{}
\fancyhead[L]{Name der Arbeit}
\fancyhead[R]{\thepage}

% Kapitelüberschrift ohne "Chapter"
\titleformat{\chapter}[hang]
  {\normalfont\huge\bfseries} % Format der Schrift
  {\thechapter.} % Nummerierung des Kapitels mit Punkt
  {1em} % Abstand zwischen Nummer und Titel
  {}

\usepackage{caption} % Paket für Beschriftungen

% Beschriftungsformat für Tabellen anpassen
\captionsetup[table]{
    labelformat=simple,
    labelsep=colon, % Trennt "Tabelle Nr" und Titel mit einem Doppelpunkt
    name=Tabelle % Festlegen des Präfixes für die Beschriftung
}

% Nummerierung der Tabellen ohne Abschnittsnummer
\renewcommand{\thetable}{\arabic{table}} % Tabellen nur fortlaufend nummerieren

% Beschriftungsformat für Abbildungen anpassen
\captionsetup[figure]{
    labelformat=simple,
    labelsep=colon, % Trennt "Abbildung Nr" und Titel mit einem Doppelpunkt
    name=Abbildung % Festlegen des Präfixes für die Beschriftung
}

% Nummerierung der Abbildungen ohne Abschnittsnummer
\renewcommand{\thefigure}{\arabic{figure}} % Abbildungen nur fortlaufend nummerieren

\usepackage{tocloft} % Für Anpassungen im Inhalts-, Abbildungs- und Tabellenverzeichnis

% Anpassung des Tabellenverzeichnisses
\renewcommand{\cfttablename}{Tabelle} % Präfix "Tabelle" hinzufügen
\renewcommand{\cfttabpresnum}{\cfttablename\enspace} % Präfix vor der Nummer
\renewcommand{\cfttabaftersnum}{:} % Doppelpunkt nach der Nummer
\setlength{\cfttabnumwidth}{6em} % Erhöhte Breite für "Tabelle Nr" im Verzeichnis

% Anpassung des Abbildungsverzeichnisses
\renewcommand{\cftfigurename}{Abbildung} % Präfix "Abbildung" hinzufügen
\renewcommand{\cftfigpresnum}{\cftfigurename\enspace} % Präfix vor der Nummer
\renewcommand{\cftfigaftersnum}{:} % Doppelpunkt nach der Nummer
\setlength{\cftfignumwidth}{6em} % Erhöhte Breite für "Abbildung Nr" im Verzeichnis

% Anpassung der Verzeichnisüberschriften
\renewcommand{\contentsname}{Inhaltsverzeichnis}
\renewcommand{\listfigurename}{Abbildungsverzeichnis}
\renewcommand{\listtablename}{Tabellenverzeichnis}
\renewcommand{\bibname}{Literaturverzeichnis}

% Variablen für die Arbeit
\newcommand{\ThesisTitle}{Titel des Papers}
\newcommand{\Module}{Lehrveranstaltung}
\newcommand{\AuthorName}{Max Mustermann}
\newcommand{\MatriculationNumber}{123456}
\newcommand{\Supervisor}{Prof. Dr. Beispiel}
\newcommand{\SubmissionDate}{01.01.2024} % Lädt die Variablen aus der Datei variables.tex

\begin{document}

% Sperrvermerk
\input{A_Template/02_sperrvermerk}
\thispagestyle{empty}

% Deckblatt
\begin{titlepage}
    \raggedright % Flattersatz nur für das Deckblatt
    \input{A_Template/01_deckblatt} % Einbindung der ausgelagerten Datei
\end{titlepage}
\thispagestyle{empty}

% Einfügen der Eidesstattlichen Erklärung
\input{A_Template/03_eidesstattliche_erklaerung}
\thispagestyle{empty}

% Einfügen der Kurzfassung/Abstract
\newpage
\chapter*{Kurzfassung/Abstract}

\section*{Kurzfassung}
Diese Masterarbeit behandelt das Thema XYZ und analysiert die Aspekte ABC basierend auf theoretischen Grundlagen und empirischen Erkenntnissen. Ziel der Arbeit ist es, Handlungsempfehlungen abzuleiten, die für DEF von Relevanz sind. Die Ergebnisse zeigen, dass GHI einen entscheidenden Einfluss auf JKL haben und durch geeignete Maßnahmen optimiert werden können.

Die Arbeit liefert einen Beitrag zur aktuellen Forschung im Bereich MNO und kann als Grundlage für weitere Studien dienen.

%\cite{LABEL1,LABEL2,NORM1}

\section*{Abstract}
This master's thesis explores the topic of XYZ and analyzes aspects of ABC based on theoretical foundations and empirical findings. The aim of the study is to derive practical recommendations relevant for DEF. The results demonstrate that GHI significantly impacts JKL and can be optimized through appropriate measures.

The work contributes to current research in the field of MNO and serves as a basis for future studies.

%\cite{LABEL1,LABEL2,NORM1}
\thispagestyle{empty}

\newpage
\thispagestyle{empty}
\cleardoublepage

% Inhaltsverzeichnis mit Seitennummerierung starten
\pagenumbering{Roman} % Römische Zahlen starten
\setcounter{page}{1}  % Startet die Nummerierung bei 1
\tableofcontents
\newpage

% Abbildungsverzeichnis
\listoffigures
\addcontentsline{toc}{chapter}{\listfigurename} % Zum Inhaltsverzeichnis hinzufügen
\newpage

% Tabellenverzeichnis
\listoftables
\addcontentsline{toc}{chapter}{\listtablename} % Zum Inhaltsverzeichnis hinzufügen
\newpage

% Abkürzungsverzeichnis
\chapter*{Abkürzungsverzeichnis}
\addcontentsline{toc}{chapter}{Abkürzungsverzeichnis}
\begin{description}
    \item[AI] Artificial Intelligence
    \item[CRM] Customer Relationship Management
\end{description}
\newpage

% Arabische Seitennummerierung für den Hauptteil
\pagenumbering{arabic}
\setcounter{page}{1} % Startet die arabische Nummerierung bei 1

%######### KAPITEL EINFÜGEN ##############
% Beispielkapitel
\chapter{Einleitung}
Dies ist der Einleitungstext. Ziel der Arbeit ist es...

\section{Hintergrund}
Lorem ipsum dolor sit amet, consetetur sadipscing elitr, sed diam nonumy eirmod tempor invidunt ut labore et dolore magna aliquyam erat, sed diam voluptua. At vero eos et accusam et justo duo dolores et ea rebum. Stet clita kasd gubergren, no sea takimata sanctus est Lorem ipsum dolor sit amet. Lorem ipsum dolor sit amet, consetetur sadipscing elitr, sed diam nonumy eirmod tempor invidunt ut labore et dolore magna aliquyam erat, sed diam voluptua. At vero eos et accusam et justo duo dolores et ea rebum. Stet clita kasd gubergren, no sea takimata sanctus est Lorem ipsum dolor sit amet.
\cite{Muster.2024}

\subsection{Relevanz der Thematik}
Warum ist dies wichtig...
\chapter{Theorie}
\section{Strategisches Management}
Einleitung in das Thema strategisches Management.

% Abbildung
\begin{figure}[h!]
    \centering
   
    \caption{Beispielhafte Abbildung mit Titel.}
    \label{fig:beispiel}
\end{figure}

% Tabelle
\begin{table}[h!]
    \centering
    \caption{Beispieltabelle mit Titel.}
    \label{tab:beispieltabelle}
    \begin{tabular}{|c|c|}
        \hline
        Spalte 1 & Spalte 2 \\
        \hline
        Wert 1 & Wert 2 \\
        \hline
    \end{tabular}
\end{table}
%###### KAPITEL EINFÜGEN ENDE ############

% Erklärung zu generativer KI
\chapter*{Erklärung zu generativer KI und KI-gestützten Technologien}
\addcontentsline{toc}{chapter}{Erklärung zu generativer KI und KI-gestützten Technologien}
\input{A_Template/99_author_contribution} % Einbindung der ausgelagerten Datei

% Literaturverzeichnis
\addcontentsline{toc}{chapter}{\bibname}
\bibliographystyle{apalike}
\bibliography{B_Literatur/100_literatur} % Pfad zur Literaturdatenbank

% Anhang
\appendix
\chapter{Anhang}
Zusätzliche Materialien.

\end{document}