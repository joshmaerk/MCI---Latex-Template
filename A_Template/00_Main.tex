\documentclass[11pt,a4paper]{report}
\usepackage[utf8]{inputenc}
\usepackage[left=3.5cm, right=2.5cm, top=2.5cm, bottom=2.5cm]{geometry}
\usepackage{setspace}
\usepackage{titlesec}
\usepackage{fancyhdr}
\usepackage{graphicx}
\usepackage{caption}
\usepackage{footmisc}
\usepackage{hyperref}
\usepackage{etoolbox}

% Schriftart und Zeilenabstand
\renewcommand{\familydefault}{\sfdefault} % Schriftart (Arial-ähnlich)
\setlength{\parindent}{0pt} % Kein Absatzeinzug
\setlength{\parskip}{0.5em} % Abstand zwischen Absätzen
\onehalfspacing % Zeilenabstand 1,5

% Überschriftenformate anpassen: Gleiche Schriftgröße wie der Text
\titleformat{\chapter}%[hang]
  {\normalfont\bfseries\normalsize} % Normale Textgröße (normalsize) für Kapitelüberschrift
  {\thechapter.} % Kapitelnummerierung mit Punkt
  {1em} % Abstand zwischen Nummer und Titel
  {}

\titleformat{\section}
  {\normalfont\bfseries\normalsize} % Normale Textgröße (normalsize) für Abschnittsüberschrift
  {\thesection.} % Abschnittsnummerierung mit Punkt
  {1em} % Abstand zwischen Nummer und Titel
  {}

\titleformat{\subsection}
  {\normalfont\bfseries\normalsize} % Normale Textgröße (normalsize) für Unterabschnittsüberschrift
  {\thesubsection.} % Unterabschnittsnummerierung mit Punkt
  {1em} % Abstand zwischen Nummer und Titel
  {}

% Fußnotenformat
\renewcommand{\footnotelayout}{\setstretch{1}} % Zeilenabstand in Fußnoten

% Kopf-/Fußzeilen
\pagestyle{fancy}
\fancyhf{}
\fancyhead[L]{Name der Arbeit}
\fancyhead[R]{\thepage}

% Beschriftungsformat für Tabellen anpassen
\captionsetup[table]{
    labelformat=simple,
    labelsep=colon, % Trennt "Tabelle Nr" und Titel mit einem Doppelpunkt
    name=Tabelle % Festlegen des Präfixes für die Beschriftung
}

% Nummerierung der Tabellen ohne Abschnittsnummer
\renewcommand{\thetable}{\arabic{table}} % Tabellen nur fortlaufend nummerieren

% Beschriftungsformat für Abbildungen anpassen
\captionsetup[figure]{
    labelformat=simple,
    labelsep=colon, % Trennt "Abbildung Nr" und Titel mit einem Doppelpunkt
    name=Abbildung % Festlegen des Präfixes für die Beschriftung
}

% Nummerierung der Abbildungen ohne Abschnittsnummer
\renewcommand{\thefigure}{\arabic{figure}} % Abbildungen nur fortlaufend nummerieren

\usepackage{tocloft} % Für Anpassungen im Inhalts-, Abbildungs- und Tabellenverzeichnis

% Anpassung des Tabellenverzeichnisses
\renewcommand{\cfttablename}{Tabelle} % Präfix "Tabelle" hinzufügen
\renewcommand{\cfttabpresnum}{\cfttablename\enspace} % Präfix vor der Nummer
\renewcommand{\cfttabaftersnum}{:} % Doppelpunkt nach der Nummer
\setlength{\cfttabnumwidth}{6em} % Erhöhte Breite für "Tabelle Nr" im Verzeichnis

% Anpassung des Abbildungsverzeichnisses
\renewcommand{\cftfigurename}{Abbildung} % Präfix "Abbildung" hinzufügen
\renewcommand{\cftfigpresnum}{\cftfigurename\enspace} % Präfix vor der Nummer
\renewcommand{\cftfigaftersnum}{:} % Doppelpunkt nach der Nummer
\setlength{\cftfignumwidth}{6em} % Erhöhte Breite für "Abbildung Nr" im Verzeichnis

% Anpassung der Verzeichnisüberschriften
\renewcommand{\contentsname}{Inhaltsverzeichnis}
\renewcommand{\listfigurename}{Abbildungsverzeichnis}
\renewcommand{\listtablename}{Tabellenverzeichnis}
\renewcommand{\bibname}{Literaturverzeichnis}

\usepackage{tocloft} % Für Anpassungen der Verzeichnisüberschriften

% Schriftgröße für Verzeichnisüberschriften auf 11pt setzen
\renewcommand{\cfttoctitlefont}{\normalfont\normalsize\bfseries} % Inhaltsverzeichnis
\renewcommand{\cftloftitlefont}{\normalfont\normalsize\bfseries} % Abbildungsverzeichnis
\renewcommand{\cftlottitlefont}{\normalfont\normalsize\bfseries} % Tabellenverzeichnis

% Einheitliches Layout für alle Seiten
\fancyhf{} % Setzt Kopf- und Fußzeilen zurück
\fancyfoot[R]{\thepage} % Seitennummer rechts unten
\renewcommand{\headrulewidth}{0pt} % Entfernt die Linie in der Kopfzeile
\renewcommand{\footrulewidth}{0pt} % Entfernt die Linie in der Fußzeile

% Kapitelüberschrift: Gleicher Abstand nach unten wie bei Abschnittsüberschriften
\titlespacing{\chapter}{0pt}{0.5cm}{0.5cm} % Abstand oben und unten 0.5cm

% Abschnittsüberschrift: Gleicher Abstand wie Kapitelüberschrift
\titlespacing{\section}{0pt}{0.5cm}{0.5cm}

% Unterabschnittsüberschrift: Optional anpassen, falls nötig
\titlespacing{\subsection}{0pt}{0.5cm}{0.5cm}

\usepackage[style=apa,backend=biber,lang=ngerman]{biblatex} % APA-Format und deutsche Sprache
\addbibresource{B_Literatur/100_literatur.bib} % Pfad zur Literaturdatenbank