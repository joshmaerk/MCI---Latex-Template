In der vorliegenden Praxisarbeit wurde die Unterstützung von generativen KI-Systemen
als Hilfsmittel für die Recherche, die Strukturierung von Inhalten und die Formulierung
von Texten in Anspruch genommen. Der Autor dieser Arbeit war verantwortlich für die
Konzeption, Planung und Durchführung der Forschungsarbeit, einschließlich der Definition
der Forschungsfragen, der Analyse der Daten und der Interpretation der Ergebnisse.

Generative KI-Systeme wurden insbesondere zur Unterstützung in den folgenden Bereichen
eingesetzt:
\begin{itemize}
    \item \textbf{Recherche und Literatursuche:} Generative KI-Systeme halfen bei der Identifikation relevanter Literaturquellen und der Formulierung von Suchstrategien.
    \item \textbf{Ideenentwicklung und Strukturierung:} Die Modelle wurden genutzt, um verschiedene Perspektiven und Ansätze zur Bearbeitung der Themenstellung zu entwickeln und die Struktur der Arbeit zu optimieren.
    \item \textbf{Textformulierung und -paraphrasierung:} Generative KI-Systeme unterstützten bei der Formulierung und Paraphrasierung von Texten, um präzise und verständliche Darstellungen der Ergebnisse und Diskussionen zu gewährleisten.
\end{itemize}

Die finale Entscheidung über die Einbindung von Inhalten, die Interpretation der Ergebnisse
sowie die Schlussfolgerungen der Arbeit oblagen jedoch ausschließlich dem Autor.
Alle durch Generative KI-Systeme generierten Inhalte wurden sorgfältig geprüft, angepasst
und in den Kontext der spezifischen Fragestellung dieser Arbeit integriert. Die Verantwortung
für den gesamten Inhalt und die wissenschaftliche Integrität dieser Arbeit
liegt beim Autor.